\vsection
[Список из перестановок действительно является таковым]
{Тип Unicalized}

\parhref{https://github.com/boris-a-zolotov \\
	/Computational-complexity-of-quickest-sort \\
	/blob/master/final\_length.agda}

\abz Пусть алгоритм сортировки расставляет в свои листья перестановки (длины $n$) из списка $l$, на котором он определён. Для доказательства некоторых важных свойств алгоритмов нам потребуется явное доказательство того, что все элементы $l$ являются перестановками в смысле математического определения (как и обещано в предыдущем параграфе) — то есть

$$\forall~ k \in \{1 \ldots (\mathrm{length}\ l) - 1\}
\ \ \forall~ i,j \in \{1 \ldots (n-1)\}$$

\vspace{-0.4cm}
$$i \ne j \ \ \Longrightarrow
\ \ (l_k)_i \ne (l_k)_j.$$

\abz Этому утверждению (опять же, посредством изоморфизма Карри—Ховарда) будет соответствовать тип {\tt Unicalized}, определённый следующим образом:

\begin{verbatim}
data Unicalized : {n : ℕ} →
    List (BorderedNlist n) → Set
where
    u-nil : {n : ℕ} → Unicalized {n} []
    _u-cons_ : {n : ℕ} →
        {l : List (BorderedNlist n)} → {x : BorderedNlist n} →
        ((i j : Fin n) → (i ≢ j) → nth x i ≢ nth x j) →
        Unicalized l → Unicalized (x :: l)
\end{verbatim}

\abz Элемент типа {\tt Unicalized l} на самом деле является списком, элементы которого соответствуют элементам {\tt l}. Каждый элемент этого списка есть доказательство того, что соответствующий ему элемент {\tt l} — перестановка.

\abz Несложно доказать следующую теорему:

\newthm{projUnic}{{\bf (projUnic)\ \ \ }Если список $l_1$ является подпоследовательностью (подсписком) списка $l_0$, то $\mathtt{Unicalized}\ l_0$ влечёт $\mathtt{Unicalized}\ l_1$.}

Очевидно, что если к любому списку применить функцию {\tt filter} по любому предикату, то мы получим его подсписок. На этом основании из теоремы~\ref{projUnic} очевидно следует

\newthm{filterUnic}{
{\bf (filterUnic)\ \ \ }Если для списка $l$, состоящего из перестановок длины $n$, верно $\mathtt{Unicalized}\ l$, то
$$\forall~ i,j \in \{1 \ldots (n-1)\} \text{\ \ имеет место\ \ }
	\mathtt{Unicalized}\ (\mathtt{cmpFilter}\ i\ j\ l).$$}

Приведём пример утверждения, для доказательства которого необходимо наличие {\tt Unicalized} для исходного списка — и которое, легко понять, не верно в противном случае:

\def\lth{\mathrm{length}\ }

\newthm{FILTERLENGTH}{
{\bf (FILTERLENGTH)\ \ \ }Если для списка $l$ доказано $\mathtt{Unicalized}\ l$, то для любых $i,j \in \{1 \ldots (n-1)\}$, $i \ne j$ выполнено
$$\lth l\ \ \le\ \ 
	\lth (\mathtt{cmpFilter}\ i\ j\ l) +
	\lth (\mathtt{cmpFilter}\ j\ i\ l).$$}

С доказательствами этих и других утверждений читатель может ознакомиться в исходном коде.
\vsection
[Определение перестановки]
{Перестановки}

\parhref{https://github.com/boris-a-zolotov \\
	/Computational-complexity-of-quickest-sort \\
	/blob/master/permutation\_definition.agda}

\abz В дальнейшем нам пригодится понятие перестановки: перестановка длины $n$ — это список из $n$ элементов, в котором все элементы — различные целые числа от 0 до $n-1$. Мы для простоты не будем включать в определение для Agda требование того, что все элементы в списке обязаны быть различными, а добавим его в явном виде позже, когда это свойство нам понадобится.

\abz Для начала определим тип «список длины $n$»:

\begin{verbatim}
data Nlist (A : Set) : ℕ → Set where
    «» : Nlist A 0
    _—_ : {n : ℕ} → A → Nlist A n → Nlist A (suc n)
\end{verbatim}

После этого скажем, что нужный нам тип «перестановка» — список длины $n$, составленный из целых чисел от $0$ до $n-1$:

\begin{verbatim}
BorderedNlist : ℕ → Set
BorderedNlist n = Nlist (Fin n) n
\end{verbatim}

Тип {\tt Fin} $n$ (целые числа, строго меньшие $n$) строится в соответствии с аксиоматикой Пеано: элемент {\tt zerof} — ноль — принадлежит типу {\tt Fin} $n$ при всяком $n \ge 1$, а конструктор {\tt sucf} — «следующее число» — делает из элемента типа {\tt Fin} $n$ элемент типа {\tt Fin} $n+1$. Таким образом, единица в типе {\tt Fin} 3 \linebreak будет выглядеть как {\tt sucf} {\tt zerof}, где {\tt zerof} принадлежит типу {\tt Fin} 2.

\abz Тогда, например, перестановка $[0,2,1]$ будет теперь выглядеть так:

\begin{verbatim}
p021 : BorderedNlist 3
p021 = zerof — ((sucf (sucf zerof)) — ((sucf zerof) — «»))
\end{verbatim}

Для типа «перестановка» можно определить функцию {\tt nth}, возвращающую по номеру элемент в перестановке под этим номером. Также нам пригодится предикат {\tt listCmp} {\tt i} {\tt j} {\tt p}, отвечающий на вопрос: «верно ли, что $i$--ый элемент перестановки {\tt p} меньше, чем её $j$--ый элемент?». С тем, как описаны эти функции, читатель может ознакомиться в исходном коде.

\abz Имея предикат, можно определить соответствующую функцию высшего порядка:

\begin{verbatim}
cmpFilter : {n : ℕ} → (i j : Fin n) →
    (l : List (BorderedNlist n)) → List (BorderedNlist n)
cmpFilter {n} i j l = filter (listCmp {n} i j) l
\end{verbatim}
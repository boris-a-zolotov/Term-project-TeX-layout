\vsection
[Формализация понятия алгоритма сортировки]
{Алгоритм сортировки и его характеристики}

\subsection
[Алгоритм сортировки в Agda]
{Алгоритм}

\parhref{https://github.com/boris-a-zolotov \\
	/Computational-complexity-of-quickest-sort \\
	/blob/master/algorithm.agda}

\abz В первом разделе мы определили алгоритм сортировки. На Agda это определение будет выглядеть следующим образом:

\begin{verbatim}
data Alg : {n : ℕ} → List (BorderedNlist n) → Set where
    leaf : {n : ℕ} → (b : BorderedNlist n) → Alg (b :: [])
    node : {n : ℕ} → {l : List (BorderedNlist n)} →
        (i j : Fin n) → (i ≢ j) →
        Alg (cmpFilter i j l) → Alg (cmpFilter j i l) → Alg l
\end{verbatim}

Действительно, всякий алгоритм сортировки — это либо «лист» (когда мы наверняка нашли единственную перестановку), либо «узел»: когда мы сравниваем элементы на двух фиксированных позициях в каждой из перестановок, и для каждого из подсписков (где $(l_k)_i < (l_k)_j$, и где $(l_k)_i \not < (l_k)_j$) у нас есть алгоритм сортировки.

\abz С примерами описания алгоритмов сортировки читатель может ознакомиться в исходном коде.

\subsection
[Глубина и количество листьев алгоритма сортировки]
{Глубина и количество листьев}

\parhref{https://github.com/boris-a-zolotov \\
	/Computational-complexity-of-quickest-sort \\
	/blob/master/alg2n.agda}

\abz Глубина алгоритма сортировки (она также была рассмотрена нами в первом разделе) определяется в Agda следующим образом:

\begin{verbatim}
depth : {n : ℕ} →
    {l : List (BorderedNlist n)} → Alg l → ℕ
depth (leaf b) = 0
depth (node i j p a1 a2) = suc ((depth a1) ⊔ (depth a2))
\end{verbatim}

Количество листьев для алгоритма вводится аналогично:

\begin{verbatim}
leafnum : {n : ℕ} →
    {l : List (BorderedNlist n)} → Alg l → ℕ
leafnum (leaf b) = 1
leafnum (node i j p a1 a2) = (leafnum a1) + (leafnum a2)
\end{verbatim}

Как уже было сказано, количество листьев алгоритма определяет, сколько перестановок он может обработать. Это и утверждает следующий факт, для доказательства которого потребовалась теорема \ref{FILTERLENGTH}:

\newthm{lenLeafNum}{
{\bf (lenLeafNum)\ \ \ }Для любого алгоритма $A$, принадлежащего типу $\mathtt{Alg}\ l$, при условии $\mathtt{Unicalized}\ l$ выполнено
$$\lth l\ \le\ \mathtt{leafnum}\ A.$$
}

\subsection
[Связь между сложностью алгоритма и длиной его списка]
{Связь между ними}

С использованием теоремы \ref{lenLeafNum} доказан крайне полезный в дальнейшем аналог теоремы \ref{optimalsort}:

\newthm{len2depth}{{\bf (len2depth)\ \ \ }
$$\forall~l\ \ \ \forall~A \in \mathtt{Alg}\ l$$
\vspace{-0.4cm}
$$\mathtt{Unicalized}\ l\ \ \Longrightarrow\ \ \lth l\ \le \ 2^{\mathtt{depth}\,A}.$$
\vspace{-0.4cm}}

Доказательства всех упомянутых теорем приведены в исходном коде.
\vsection
[Примеры доказательств в Agda: свойства натуральных чисел]
{Равенство и неравенство натуральных чисел}

\parhref{github.com/boris-a-zolotov\\
	/Computational-complexity-of-quickest-sort\\
	/blob/master/equation\_properties.agda}

\abz Если читатель не знаком близко с техникой доказательств в таких языках, как Agda, — в этом разделе мы приводим примеры формальных доказательств простых свойств натуральных чисел. В частности, мы докажем, что равенство является отношением эквивалентности и $\le$ является отношением порядка. Также будут доказаны некоторые другие естественные свойства натуральных чисел, которые в Agda неочевидны и нуждаются в формальном доказательстве.

\abz За равенство элементов в Agda отвечает тип «$a \equiv b$», соответствующий посредством изоморфизма Карри—Ховарда утверждению «выражения для $a$ и для $b$ одновременно $\beta$-редуцируются к одной и той же нормальной форме».

\abz Тип $a \equiv b$ в зависимости от $a$ и $b$ может быть либо пуст, либо населён единственным элементом {\tt refl}. Если Agda может, просто редуцировав выражения для $a$ и для $b$, получить их общую нормальную форму (как, например, в случае $a = 2+1$, $b = 3$), то утверждение $a \equiv b$ можно посчитать «очевидным», и успешно верифицируется доказательство:

\begin{verbatim}
proof : 2 + 1 ≡ 3
proof = refl
\end{verbatim}

\abz В более сложных случаях (когда, например, выражения для $a$ и для $b$ зависят от других параметров) требуется, соответственно, более развёрнутое доказательство. Но при каждом конкретном значении параметров любое такое сложное доказательство окажется эквивалентно {\tt refl}-у: ведь если два выражения равны, то их можно $\beta$-редуцировать к одной и той же форме.

\begin{stat}
\label{eq_comm}
	{\bf (eq\_comm)\ \ \ }Отношение равенства симметрично: для всех $a$, $b$ выполнено\ \ 
	$a \equiv b \Longrightarrow b \equiv a$.
\end{stat}

\begin{proof}
	Нам нужно по любому доказательству равенства $a \equiv b$ научиться строить доказательство равенства $b \equiv a$. Доказательством равенства может быть только {\tt refl}, поэтому единственный вариант для нас — по {\tt refl}-у возвращать его же. Получается:

\begin{verbatim}
eq_comm : ∀ {P : Set} {m n : P} → (m ≡ n) → (n ≡ m)
eq_comm refl = refl
\end{verbatim}

\noindent Это и является утверждением и доказательством данной теоремы.\end{proof}

\begin{stat}
\label{eq_trans}
	{\bf (eq\_trans)\ \ \ }Отношение равенства транзитивно: для всех $a$, $b$, $c$ выполнено\ \ 
	$a \equiv b$, $b \equiv c$ $\Longrightarrow$ $a \equiv c$.
\end{stat}

Данная теорема доказывается аналогично: единственно возможные доказательства, которые могут поступить нам на вход — это {\tt refl}-ы, вернуть в таком случае мы тоже должны будем просто {\tt refl}.

\abz Абсолютно аналогично доказываются две следующие теоремы:

\begin{stat}
\label{eq_cong}
	{\bf (eq\_cong)\ \ \ }$\forall~ P, Q\ \ \forall~ a, b \in P\ \ \forall~ f \colon P \rar Q$
	$$a \equiv b\ \ \Longrightarrow\ \ f(a) \equiv f(b).$$
\end{stat}

Последнее свойство также называется конгруэнтностью.

\begin{stat}
\label{preserv-+}
	{\bf (preserv-+)\ \ \ }$\forall~ m,n,x,y \in \N$
	$$m \equiv n,\ x \equiv y\ \ \Longrightarrow\ \ m+x \equiv n+y.$$
\end{stat}

Для нестрогого сравнения ($\le$) нам нужно доказать то, что оно является отношением порядка: его рефлексивность, антисимметричность и транзитивность. При этом доказательство утверждения «$a \le b$» для двух чисел $a$, $b$ может иметь один из двух видов:

\begin{itemize}
\item {\tt z≤n}, если $a=0$, $b$ — произвольное\scolon
\item {\tt s≤s p}, если $a,b > 0$, и {\tt p} — доказательство того, что $a-1 \le b-1$.
\end{itemize}

\newthm{trans_<=}{
{\bf (trans\_≤)\ \ \ }Для любых натуральных чисел $a,b,c$ выполнено $a \le b,\ b \le c \Longrightarrow a \le c$.
}

\begin{proof}
	Как обычно, по паре доказательств для $a \le b$ и $b \le c$ нам нужно построить доказательство для $a \le c$. Заметим, что если первое доказательство имеет вид {\tt z≤n}, то непременно $a = 0$ (и поэтому не превосходит любое $c$), поэтому мы можем вернуть {\tt z≤n}. Это будет базой индукции.

\ms Переход: если $a$, $b$, $c$ строго больше нуля, то первое доказательство имеет вид {\tt s≤s p}, а второе {\tt s≤s q}. Тогда рассмотрим числа $a-1$, $b-1$, $c-1$ и пару доказательств {\tt p}, {\tt q}. По индукционному предположению, мы можем построить доказательство {\tt P} того, что $a-1 \le c-1$. Тогда {\tt s≤s P} и будет доказательством факта $a \le c$.

\ms Получается:

\begin{verbatim}
trans_≤ : ∀ {a b c : ℕ} → (a ≤ b) → (b ≤ c) → (a ≤ c)
trans_≤ z≤n _ = z≤n
trans_≤ (s≤s p) (s≤s q) = s≤s (trans_≤ p q)
\end{verbatim}
\end{proof}


%%%%%%%%%%%%%%%%
%%%%%%%%%%%%%%%%

\newthm{antisymm}{
	{\bf (antisymm)\ \ \ }Отношение $\le$ антисимметрично: для любых двух натуральных чисел $a, b$ выполнено:
	$$a \le b\ \land\ b \le a\ \ \Longrightarrow\ \ a \equiv b$$
}

\begin{proof}
Рассмотрим несколько случаев:

\begin{enumerate}
\item Если $a=b=0$, то вне зависимости от конкретных доказательств того, что они не больше друг друга, доказательством утверждения $0 \equiv 0$ будет очевидно являться {\tt refl}. Его и вернём — это будет базой индукции.

\item Если ровно одно из двух чисел $a$, $b$ равно нулю, а другое — нет, то не сможет быть доказуемым хотя бы одно из пары утверждений $a \le b$, $b \le a$. Поэтому такой случай, очевидно, невозможен — в Agda в таком случае необходимо написать круглые скобки\scolon это будет верифицировано.

\item Индукционный переход: если $a>0$, $b>0$, $a\le b$ и $b \le a$, то рассмотрим числа $a-1$ и $b-1$. Для них выполнено то же самое: $a-1 \le b-1$, $b-1 \le a-1$. Но тогда мы можем доказать $a-1 \equiv b-1$. В свою очередь, по теореме \ref{eq_cong} (взяв $f(x) = x+1$) имеем $(a-1)+1 \equiv (b-1)+1$ — то есть, $a \equiv b$.
\end{enumerate}

Получаем:
\begin{verbatim}
antisymm : (m n : ℕ) → m ≤ n →
    n ≤ m → n ≡ m
antisymm 0 0 p q = refl
antisymm (suc x) 0 ()
antisymm 0 (suc x) p ()
antisymm (suc x) (suc y) (s≤s p) (s≤s q) =
    eq_cong (suc) (antisymm x y p q)
\end{verbatim}
\end{proof}

Аналогично доказывается

\newthm{noteq}{
{\bf (noteq)\ \ \ }Для любых $m,n \in \N$, если $m < n$, то не верно, что $m = n$.}

\begin{proof}
\ \begin{verbatim}
noteq : (m n : ℕ) → m < n → m ≡ n → ⊥
noteq 0 0 ()
noteq (suc x) 0 ()
noteq 0 (suc x) p ()
noteq (suc x) (suc y) (s≤s p) q =
    noteq x y p (eq_cong (pred) q)
\end{verbatim}
\end{proof}